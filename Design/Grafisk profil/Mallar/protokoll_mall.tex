
%% example.tex: Manual till den stora F-klassen(TM) med exempel.
%% Av Johan Förberg F10, 2013

%% Ladda in F-klassen.
\documentclass[protokoll]{fclass}

%% Om du vill ta med F-papperet i PDF:en digitalt skriver du istället
 %\documentclass[nofpaper]{fclass}
%% Använd detta bara när du vet att du inte kommer att behöva skriva ut,
%% och bara vill maila ut dokumentet. Skriv alltid ut på riktigt F-papper!

%% F-klassen tar hand om textkodning och annat bök åt dig. Bra va?

%% Rubrik för dokumentet.
\title{Protokoll från styrelsemöte F1--HT93 \\
       \normalsize{Alumnirummet, 
       kl. 12.15 onsdagen den 2 september 2093}}

%% Kortare rubrik som vi kan använda i sidhuvudet.
 \shorttitle{F1--HT93}
%% Ta inte med "motion angående" eller liknande. Kort och koncist.

%% Vem har författat dokumentet?
\author{\f -sektionens styrelse}
%% Ditt namn, namnet på en arbetsgrupp eller "\f -sektionens styrelse".

%% När skrevs dokumentet?
\date{\today}
%% Värdet bör alltid vara \today.

%% Vad är det för slags dokument? 
\type{Protokoll}
%% Exempel: Motion, Protokoll, Remiss, Dagordning.

%% Plats för att ladda paket. Kolla först i fclass.cls om det finns där.
 %\usepackage{myawesomepackage}

%% Här börjar dokumentet.
\begin{document}

%% Påbörja en ny paragraf med \para 

\para{O.F.M.Ö.}
Ordförande Trula Teknologsson förklarade mötet öppnat.

\para{Närvaro}

%% Närvarolistan gör man så här. Skriv \\ (radbrytning) efter varje namn
%% utom det sista.

\begin{narvaro}
	\textbf{Trula Teknologsson}\\
    \textbf{Truls Trulasson}\\
    \textbf{Teknolog Teknologsson}\\
    \textbf{Teknolog Trulsson}\\
    \textbf{Truls Teknologsson}\\
    \textbf{Trula Trulasdotter}\\
    \textbf{Truls von Technik}\\
    \textbf{Truls Trulsson}\\
    \textbf{Teknolog de Technique} \emph{fr.o.m. §7}\\
    Hilbert Älg\\
    Hella Älg
\end{narvaro}

\para{Adjungeringar}
Alla som skrivit upp sig på närvarolistan samt som inte var ständigt adjungerade eller styrelseledamöter adjungerades.

\para{Val av justerare}
Kassör Teknolog Trulsson utsågs att, jämte ordförande Trula Teknologsson, justera protokollet.

\para{Godkännandede av dagordning}
Styrelsen beslutade 
\batt
godkänna dagordningen som den låg.
\eatt


\para{Meddelanden}

Vice ordförande Teknolog Teknologsson meddelade att potatisskörden hade varit dålig i Skåne innevarande år.


\para{Beslutsuppföljning}
\begin{beslutsupf}
F12--VT92&Älghorn&Truls von Technik&F01--HT93\\ % Skriv F01--HT93 inte F1--HT93 som vanligt för att allt ska ligga i linje.
\end{beslutsupf}

\textbf{Älghorn:} Bla bla bla...

Styrelsen beslutade 
\batt
stryka punkten från beslutsuppföljningen.
\eatt


\para{Arbetsgruppsrapporter och redovisningar}
\begin{arb}
F122--VT93&Hilberts frack&Truls Trulsson&F01--HT93\\
\end{arb}

\textbf{Hilberts frack:} Bla bla bla...
\batt
flytta redovisningsdatum till F3--HT93.
\eatt



\para{Inköp av potatis}

Bla bla bla...

Styrelsen beslutade
\batt
avsätta upp till 10\,000~kronor ur kontot för styrelsen disponibelt för inköp av potatis.
\eatt


\para{Övrigt}

Inga övriga frågor lyftes.



\para{Nästa möte}

Nästa möte sattes till onsdagen den 9 september klockan 12.15 i Alumnirummet


\para{O.F.M.A.}
Ordförande Trula Teknologsson förklarade mötet avslutat.

%% Lämna lite plats mellan texten och underskrifterna.

\vspace{10mm}

%% Här börjar blocket med signaturer. Miljön ser automatiskt till att 
%% alla signaturer kommer på samma sida.

\begin{signblock}

\emph{I sektionsstyrelsens tjänst,}
%% Din post eller annan titel skriver du inom hakparentes.

\sign[Ordförande]{Trula Teknologsson}
\sign[Sekreterare]{Truls Trulasson}
\sign[Justerare]{Teknolog Trulsson}

%% Här kan man fylla på med fler namn om man vill. Miljön lägger dem 
%% automatiskt i två kolumner.
\end{signblock}

%% Slut.

%\includepdf[pages=-]{}

\end{document}